\documentclass[17pt,a4paper]{extarticle}%
\usepackage{amsfonts}
\usepackage{fancyhdr}
\usepackage{comment}
\usepackage{tikz}

\usepackage[a4paper, top=2.5cm, bottom=2.5cm, left=2.2cm, right=2.2cm]%
{geometry}
\usepackage{times}

\usepackage{amsmath}
\usepackage{changepage}
\usepackage{amssymb}
\usepackage{graphicx}%
\setcounter{MaxMatrixCols}{30}
\newtheorem{theorem}{Theorem}
\newtheorem{acknowledgement}[theorem]{Acknowledgement}
\newtheorem{algorithm}[theorem]{Algorithm}
\newtheorem{axiom}{Axiom}
\newtheorem{case}[theorem]{Case}
\newtheorem{claim}[theorem]{Claim}
\newtheorem{conclusion}[theorem]{Conclusion}
\newtheorem{condition}[theorem]{Condition}
\newtheorem{conjecture}[theorem]{Conjecture}
\newtheorem{corollary}[theorem]{Corollary}
\newtheorem{criterion}[theorem]{Criterion}
\newtheorem{definition}[theorem]{Definition}
\newtheorem{example}[theorem]{Example}
\newtheorem{exercise}[theorem]{Exercise}
\newtheorem{lemma}[theorem]{Lemma}
\newtheorem{notation}[theorem]{Notation}
\newtheorem{problem}[theorem]{Problem}
\newtheorem{proposition}[theorem]{Proposition}
\newtheorem{remark}[theorem]{Remark}
\newtheorem{solution}[theorem]{Solution}
\newtheorem{summary}[theorem]{Summary}

\newenvironment{proof}[1][Proof]{\textbf{#1.} }{\ \rule{0.5em}{0.5em}}

\newcommand{\Q}{\mathbb{Q}}
\newcommand{\R}{\mathbb{R}}
\newcommand{\C}{\mathbb{C}}
\newcommand{\Z}{\mathbb{Z}}

\newcommand\namegroup[1]{%
   \begin{minipage}[t]{0.4\textwidth}
   \vspace*{1.5cm}  % leave some space above the horizontal line
   \hrule
   \vspace{1mm} % just a bit more whitespace below the line
   \centering
   \begin{tabular}[t]{c}
   #1
   \end{tabular}
   \end{minipage}}

   % Modelo elaborado por Rafael Almeida - Confraria de alunos da engenharia elétrica CAEL.

\begin{document}
% \onehalfspacing
\begin{titlepage}
\centering
  
\begin{center}
    \includegraphics[width=0.2\textwidth]{yogam.png}
\end{center}
\begin{center}
    \includegraphics[width=0.4\textwidth]{logo_em.png}
\end{center}

\vfill

\begin{center}
    \includegraphics[width=0.7\textwidth]{gujuni.png}
\end{center}
\vspace{1cm}
\begin{large}
\textbf{Department of Physics,
Electronics and Space Sciences}
\vfill
\textbf{Project Report}\\
\textbf{M. Sc. Physics, Semester 2}
\end{large}\\[10pt]
\vfill

% \begin{flushleft}
% Relatório do Experimento 7  - Laboratório de Conversão Eletromecânica de Energia\\
% Professor André Guilherme Peixoto Alves\\
% Grupo 2\\
% Alunos:\\ 
%     \hspace{1cm}Gabriela Lemos\\
%     \hspace{1cm}Renan Pinho\\
%     \hspace{1cm}Rafael Almeida\\
%     \hspace{1cm}Gabriel Fávero\\

% \end{flushleft}

\vfill

Gujarat University\\[2.5pt]
March 25, 2023
\end{titlepage}

% \title{MATH 412 True/False Questions}
% \author{YOUR NAME HERE}
% \date{\today}
% \maketitle




  
  




\end{document}
